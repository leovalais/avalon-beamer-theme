\documentclass[aspectratio=169]{beamer}
\usepackage[utf8]{inputenc}
\usepackage[T1]{fontenc}
\usepackage{graphicx}
\usepackage{tikz}
\usepackage{minted}
\usepackage{appendixnumberbeamer}
\usepackage{hyperref}
% \usepackage[texcoord,grid,gridunit=mm,gridcolor=red!20,subgridcolor=gray!10]{eso-pic}

\usemintedstyle{colorful}

\usetheme{avalon}
\def\avalonprogressbar{1}
\def\avalondarkmintedstyle{zenburn}

\title{Avalon theme}
\subtitle{A modern theme for beamer}
\date{\it{}Somewhere in time \footnotesize\faMusic}
\author{Léo Valais}
\institute{Institute}

\titlegraphic{
  \vspace*{-1em}
  \begin{tikzpicture}
    \draw[iOS-orange] (0,0) circle (1cm) node {\color{iOS-blue}Your logo};
  \end{tikzpicture}
}


\begin{document}

\begin{frame}
\titlepage{}
\end{frame}

\begin{frame}
  \frametitle{TOC}
  \tableofcontents
\end{frame}

\section{Introduction}

\begin{frame}
  \frametitle{About Avalon}
  Avalon is modern beamer theme inspired of the \texttt{metropolis} theme which
  can be found at: \textcolor{iOS-tealblue}{\href{https://github.com/matze/mtheme}{https://github.com/matze/mtheme}}.

  It redefines some beamer templates and provides some customization options in
  addition to beamer's ones.
\end{frame}

\begin{frame}
  \frametitle{There Is No Largest Prime Number}
  \framesubtitle{The proof uses \textit{reductio ad absurdum}.}
  \begin{theorem}
    There is no largest prime number.
  \end{theorem}

  \begin{enumerate}
  \item Suppose $p$ were the largest prime number.
  \item Let $q$ be the product of the first $p$ numbers.
  \item Then $q+1$ is not divisible by any of them.
  \item But $q + 1$ is greater than $1$, \alert{thus divisible by some prime
      number} not in the first $p$ numbers.
  \end{enumerate}
\end{frame}

\section{Beamer material}
\subsection{Colors}
\begin{frame}[fragile]
  \frametitle{Colors}

  \newcommand{\col}[1]{\texttt{\textcolor{#1}{#1}}}
  Avalon defines 8 iOS colors which are extensively used as default colors:
  \col{iOS-red}, \col{iOS-orange}, \col{iOS-yellow}, \col{iOS-green},
  \col{iOS-tealblue}, \col{iOS-blue}, \col{iOS-purple} and \col{iOS-pink}.

  Many other default colors are also defined in
  \texttt{beamercolorthemeavalon.sty}.

  Beamer colors are still customizable using \texttt{setbeamercolor}.
\end{frame}

\subsection{Lists}
\begin{frame}
  \frametitle{Lists}
  \framesubtitle{\texttt{itemize} and \texttt{enumerate}}
  \begin{columns}
    \column{1em}
    \column{.5\paperwidth}
    \begin{itemize}
    \item one
    \item two
      \begin{itemize}
      \item 1
      \item 2
        \begin{itemize}
        \item I
        \item II
        \item III
        \end{itemize}
      \item 3
      \end{itemize}
    \item three
    \end{itemize}

    \column{.5\paperwidth}
    \begin{enumerate}
    \item one
    \item two
      \begin{enumerate}
      \item un
      \item deux
        \begin{enumerate}
        \item I
        \item II
        \item III
        \end{enumerate}
      \item trois
      \end{enumerate}
    \item three
    \end{enumerate}
  \end{columns}
\end{frame}

\subsection{Blocks}
\begin{frame}[fragile]
  \frametitle{Blocks}
  \begin{columns}
    \column{.45\textwidth}
    \begin{block}{A block}
      Some text.
    \end{block}

    \begin{alertblock}{Pay attention!}
      I mean, really\dots
    \end{alertblock}

    \begin{theorem}
      $\forall x, \forall n, x = 2 \cdot n \Rightarrow x = (3 - 1) \cdot n$
    \end{theorem}

    \begin{exampleblock}{Meaningful example}
      $2 = 3 - 1$
    \end{exampleblock}

    \begin{block}[iOS-orange]{}
      Block without a title.

      Better looking with at least two lines of content.
    \end{block}

    \column{.55\textwidth}
    \begin{block}[pink]{Change the color!}
      \texttt{block}'s default color is beamer color \texttt{block title.fg}.

      But it can be changed
      \begin{itemize}
      \item globally:
\begin{minted}{latex}
\setbeamercolor{block title}{
  fg=your color
}
\end{minted}
      \item or for one specific block:
\begin{minted}{latex}
\begin{block}[your color]{title}
  content
\end{block}
\end{minted}
      \end{itemize}
    \end{block}

  \end{columns}
\end{frame}

\makeatletter
\makeatother
\section{Avalon-specific environments}
\subsection{Standout}
\begin{frame}[standout]
  \frametitle{Standout frames}
  This is a \texttt{standoutframe}. Background color and text colors are inverted.
  \texttt{itemize}, \texttt{enumerate} and \texttt{block title} colors are not.
  However, the default colors for these \emph{still look nice} with this background.

  \begin{block}{}
    To create a standout frame, use
% \begin{verbatim}
% \begin{frame}[standout]
% \end{verbatim}
  \end{block}

  \begin{itemize}
  \item If \texttt{minted} is imported and a dark theme is defined, it is used in this
    frame.
    \begin{exampleblock}{Dark theme example}
% \begin{minted}{c}
% #define X (.0f)
% float x = (3 - 1) * X; // it reminds me some theorem..
% \end{minted}
      TODO: insert some C here.
    \end{exampleblock}
  \end{itemize}

  This kind of frame is great to catch the audience's attention.

  % \bigskip{}
  % \begin{block}[iOS-yellow]{Note}
  %   \texttt{standoutframe}s cannot have reversed frametitles (yet.)
  % \end{block}
\end{frame}



\subsection{Blueprints}
\subsubsection{Demo}
\begin{blueprintframe}
  \frametitle{Blueprint}
  hello
\end{blueprintframe}


\begin{frame}[fragile,t]
  \frametitle{TEST}
  \begin{block}{block}
    content
  \end{block}
\end{frame}


\section{Any questions?}
% \begin{standoutframe}
%   \begin{center}
%     \scalebox{1.5}[1.5]{\Huge \bfseries Any questions?}
%   \end{center}
% \end{standoutframe}

% \appendix{}

% \begin{frame}
%   \nocite{*}
%   \bibliographystyle{plain}
%   \bibliography{example}
% \end{frame}

% \begin{frame}[fragile]
%   \frametitle{Appendix}
%   To add an appendix frame, include:

% \begin{verbatim}
% \usepackage{appendixnumberbeamer}
% \end{verbatim}

%   and in your \texttt{document} place between two \texttt{frames}:

% \begin{verbatim}
% \appendix{}
% \end{verbatim}
% \end{frame}

% \begin{frame}[fragile]
%   \frametitle{Boxes}
%   \begin{alertblock}{Fuyez!}
%     Pauvres fous\ldots
%   \end{alertblock}
%   \begin{exampleblock}{Example}
% \begin{minted}{ocaml}
% let rec fact = function
%   | 0 -> 1
%   | x -> x * (fact (x - 1));;
% \end{minted}
%   \end{exampleblock}
%   \begin{block}{Version using \texttt{if}}
%  \begin{minted}{ocaml}
% let rec fact x =
%   if x = 0 then
%     1
%   else
%     x * (fact (x - 1));;
% \end{minted}
%   \end{block}
% \end{frame}
\end{document}
